\clearpage
\section{A.2: HABERLEŞME }
\label{CH:AltBolum1.2}

Haberleşme sistemleri temel olarak 3 elemandan oluşur:

• Verici

	Sistemdeki verici tarafından gönderilen bilgi alıcıya iletilmesinden önce iletilecek hale getirilir (yani veriye dönüştürülür), kendi güç kapasitesine göre iletim yaptığı mesafenin değişmesi süretiyle, gerekli kodlamaları yapar, hatta gerekirse kuvvetlendirme de yapılır.  
    
• İletim Ortamı 

	Sistemler arasındaki iletim ortamı verici tarafından sinyalin iletildiği ortama verilen isimdir. Bu iletim ortamları kablolu ve kablosuz olmak üzere ikiye ayrılır. Kablolu İletim Ortamının tanımını yaparsak; verilerin iletimi yanlızca bu kabloların bağlı olduğu modüller arasında gerçekleşir. Bakır, fiber optik, bükümlü, koaksiyel  kablolar vs. Kablosuz İletim Ortamının tanımı ise; bu ortamlarda iletişim, veriler en uygun alıcıyı kullanarak TV ve radyo yayınlarındaki gibi herkes tarafından alınabilmektedir. En verimli ortamları hava, su, boşluk vs. gibi doğal ortamlardır.  
    
• Alıcı 

	Vericide toplanan bilgi, bahsi geçen iletim ortamlarından geçtikten sonra alıcıya gelir. 
