\textbf{D: SONUÇLAR }
\label{BolumSonuc}

	Yazılım ve Elektronik alanlarında yapılan yeni projeler ve devam eden çalışmaların en önemlileri insan sağlığını koruyacak türde veya global teknolojiye ayak uydurarak askeri ya da endüstriyel alanlarda gelişime açık olanlarıdır. Bu fikirle başladığım projeyi tamamlamak için amaçladığım gibi tareti kontol eden kişinin verdiği komutları veri kaybı olmadan tarete ulaşmasını sağlayabilmek vasıtasıyla tasarlanan taretin hedefe isabetli atışlarını gerçekleştirdim. Bu işlemin kullanıcı ve taret arasında kablosuz haberleşme ile sağlandığını vurgulamak gerekir.
    
Bu projeyi gerçekleştime aşamalarında birçok sıkıntıyla karşılaştım. İlk karşılaştığım sıkıntı satın almış olduğum arduino nano kartında arıza olmasıydı. Bu kartın hatalı olduğunu anlatıktan sonra yenisi temin ettim. ancak bu bozuk kartla işlem yapmaya kalkarken sanırsam hc-05 bluetooth modülümün de hasar aldığını farkettim ve onu da değiştirmek zorunda kaldım. Projede kullanılmasını uygun gördüğim FR207 hızlı diyot ve rfp30n06le mosfeti birçok yerde aradım ancak bulamadım. diyotun özellikleri 2A 100V değerlerini gösterirken mosfet ise 30A 60V değerlerini göstermekte. Ancak değindiğim gibi bu modülleri temin edemedeğimden bunların yerine BY298 hızlı diyot (2A 400V) ve ırfp250n mosfet (30A, 200V) modüllerini kullanmaya karar verdim. Bu modüller DC motorlara bağlantı kurduğundan, muhtemelen bu karar değişimden dolayı proje testi esnasında asıl büyük sıkıntıyı burada yaşadım. DC motorların güç kaçağı veya geri besleme gibi sıkıntılardan olsa gerek devreyi stabil tutamadım. bu kısım beni gerçekten çok uğraştırdı. İlk başlarda sıkıntının ne olduğunu da tam anlayamadım. En sonunda Voltaj dengesini sağlayarak devreyi stabil tutabildim.

Projenin sunumunu ayrı şekilde temin edeceğimden bu final tezine ekleme yapabileceğim kodlama, sonuç ve yorumlamalardan başka birşey kalmıyor. Zaten projenin son halini yazımdan çok videodaki anlatıma döktüm. Eminim yeterli gelecektir.

\clearpage