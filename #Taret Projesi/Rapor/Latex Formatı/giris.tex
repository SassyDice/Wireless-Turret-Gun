
\label{CH:BolumGiris}

Günümüz teknolojisinde endüstriyel, tıp ve askeri alanlar başta olmak üzere birçok alanda robotik sistemlere olan ilgi artmaktadır. Hobi olarak veya üniversite proje yarışmaları ile yapılan robotik sistemlerin yanı sıra askeri güç amaçlı, uluslararası rekabet ve ülke sınırları dahilindeki terör sorununu bitirmek amaçlı yapılan ve özellikle de AR-GE çalışmaları sonucu ortaya çıkarılan birçok savunma sanayi ürünü sunulmuştur. Tasarladığım uzaktan kontrollü taret projesinin, bir insanın yapamayacağı derecede hassasiyet elde ederek hedefe net kitlenebilme ve daha da önemlisi can kaybının azalacağına yönelik büyük hizmet sunacağı düşünülmektedir. Bu proje 3D yazıcı yardımıyla tasarlanmış taretin, kullanıcının vereceği komutları doğru uygulayabilmesi esasına dayanmaktadır. Tasarlanan taretin devrelerine bağlı durumda bulunan HC-05 Bluetooth modeli, işlevini uzaktaki bir kullanıcının vereceği komutları veri kaybı olmadan tarete ulaştırabilmesi amacıyla devreye eklenmiştir. Kullanıcı ile taret arasındaki haberleşme kablosuz olarak sağlanmıştır. Proje, bilgisayar veya mobil ile uyumlu nitelikte olacaktır.  Eğer ki bilgisayar üzerinden haberleşme sağlanıyorsa, fare üzerindeki hareketilerin veri kodları arduino uyumu için  kendi kütüphanesine bağlı kodlar vasıtasyıla compile edilir ve servo motorlara iletilir. Hem gövde hem de baş kısmında bulunan servolar sayesinde yatay ve dikey hareket kabiliyeti tarete akatarılmış olur. Plastik mermilerin atışının gerçekleşmesi ise namlu ucuna yerleştirilen 2 adet DC motorun sürekli dönüşü sayesinde ikisinin arasına temas eden mermiyi hızlı bir ivmeyle fırlatması şeklinde gerçekleşir. Tüm bu işlemlerin programlanması Arduino NANO mikrodenetleyicisi üzerinden kurulan devre sayesinde kontrol edilmiştir.

\clearpage