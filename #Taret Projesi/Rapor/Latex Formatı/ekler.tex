\textbf{EK 3. Standartlar ve Kısıtlar Formu}
\label{CH:BolumEkler3}

\begin{figure}[H]
	\includegraphics[scale=1.3]{grafik/eru.jpg}
    \label{fig:EruLogoDM} \centering Erciyes Üniversitesi Mühendislik Fakültesi Bilgisayar Mühendisliği
\end{figure} 

\textbf{STANDARTLAR VE KISITLAR FORMU}
\\ \hline 

\textbf{1. Projenizin tasarım boyutu nedir?}

 	Plastik mermi kullanacak şekilde Minimizal bir boyutta olup ergonomik olması hedeflenmektedir. Kullanımı ve gerçekleştirilmesi karmaşık olmayacak. Savunma sanayide kullanılabilir olup, birçok askeri ve manevi ihtiyacı karşılayacaktır.
    
\textbf{2. Üniversite eğitim süreci boyunca girdiğiniz derslerden edindiğiniz hangi deneyim ve bilgileri kullandınız?}

	Temel olarak Arduino programını kullanmamıza olanak sağlayan elektronik derslerinden ve gömülü sistemlerden projenin devresini oluşturmakta zorluk çekmedim. Ayrıca mikroişlemciler dersini almamın bana bu proje üzerinde araştırma kolaylığı sağladığı da aşikardır.
    
\textbf{3. Projenizde bir mühendislik sorununu kendi yöntemlerinizle formüle edip, çözüm sağladınız mı? }

	Yorumlama ve Değerlendirme kısmında bahsettiğim gibi bu kısım hakkında şuan için verebileceğim bir bilgi yok.
    
\textbf{4. Dikkate aldığınız esas kısıtlar nelerdir? }

a) Teknoloji

	Sürekli gelişen teknolojiye her zaman güncel kalmak kimseyi geride bırakmaz. Bu izlenimle projenin mevcut teknolojiye ayak uydurması ve başka projelere de ilham kaynağı olması hedeflenmektedir. Her yeni çıkan teknolojiler izlenmeli ve mümkün olan en yeni parçalarla ürünü sunmak gerekmektedir.
    
 b) Ekonomi
 
	Projenin üretim maliyeti, projenin hayata geçmesi ve zarar edilmemesi açısından oldukça önemli bir faktördür. Projede kullanılacak olan parçaların performansı ve fiyatı incelenerek fiyat/performans oranları çıkarılmaya çalışılmıştır. 


c) Çevre sorunları: 

	Proje düşük enerji tüketimi ile çevre dostu olduğundan ve çevreye verebileceği herhangi bir zararı bulunmadığı için oldukça kullanışlıdır.
    
d) Üretilebilirlik: 

	Projede kullanılan malzemelerin satış fiyatları oldukça uygundur. Böylelikle bu çaplı minimizal (aynı zamanda deneysel) bir projenin hayata geçirilmesi için büyük bir imkân sağlanıyor. Projeyi dinamik ve performanslı hale getirerek endüstriyel anlamda seri üretime geçilme imkânı sağlanmalıdır. Şuanda da birkaç benzer örneği mevcuttur.
    
e) Sürdürülebilirlik:

	Günümüz çağında Mikrodenetleyici kontrollü kablosuz sistemler, yeni gelişmekte olan bir sektördür. Mikroişlemcilerin, entegre parçaların, transistörlerin ve sensörlerin gelişmesiyle ilerde çok daha gelişeceğini tahmin etmek zor olmasa gerek. Proje kolay geliştirilebilir olduğundan bu imkana olanak sağlıyor. Ayrıca farklı proje ve sistemlerle ortak noktalarda birleştirilebilecek bir projedir. 
    
f) Sağlık: 

	Robot elin üretimi için kullanılacak olan PLA+ Filamentin bileşenleri mısır besi dokusundan elde edilen nişasta ve şeker pancarı gibi bitkisel, yenilebilen ürünlerden elde edilir. Bu nedenle doğaya hiçbir şekilde zarar vermez. Isı ile oluşabilecek sağlığa zararlı ve kötü koku salınımı yapan gazlar ortaya çıkarmaz.
    
 g) Güvenlik: 
 
	Projenin devresindeki tüm bileşenler voltaj regülatörü + trimpot ile dengelenmiş olup 5V ile çalışmaktadır ve hiçbir şekilde güvenlik açığı teşkil etmezler.

\clearpage