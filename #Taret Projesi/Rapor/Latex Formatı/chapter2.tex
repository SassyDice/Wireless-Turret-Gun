\chapter{B: ARDUİNO}
\label{IkinciBolum}

	Arduino kullanımı ilk başta pek az kişi tarafından bilinmekteydi. kökenleri ise Wiring ve Processing projelerine dayanmaktadır. Henüz programlama deneyimi olmayan kişilere programlamayı öğretebilmek amacıyla processing programlama dili ve geliştirme ortamı oluşturulmuştur. 
    Teknik bilgisi az olan insanlara bilgi besleme adına da kolaylık sağlayan arduino, çevresiyle etkileşim içinde olan interaktif nesneler oluşturmak isteyen tasarımcılar için de büyük kolaylık sağlamaktadır.
    Arduino bir Giriş/Çıkış kartı ve Processing dilinin uygulamasını da dahilinde bulunduran bir fiziksel programlama platformudur. Arduino bilgisayar üzerinde çalışan yazılımlarda kullanıldığı gibi tek başına çalışan interaktif nesneler oluşturmak için de kullanılabilir.
    
	Arduino’ nun neden bu kadar çok tercih edildiğini maddeler halinde gösterirsek:  
    
\begin{itemize}
\item Windows, Linux, Mac gibi bütün platformlarda çalışabilen geliştirme ortamı ve sürücülerinin kurulumu çok kolay olmaktadır.  
\end{itemize}

\begin{itemize}
\item Zengin bir kütüphane barındırdığından Birçok karmaşık işleme kolaylık sağlayabilmektedir.
\end{itemize}
	
\begin{itemize}
\item İçerisinde aktarılan kodlar/programlar oldukça hızlı çalışmaktadır. 
\end{itemize}

\begin{itemize}
\item Başka donanımlarla da çalışabilmesi adına ek donanımlar içermektedir. Bu karta bağlanamayan sensör çeşidi yok desek yeridir.
\end{itemize}
    
\begin{itemize}
\item Benzer diğer devre kartlarına kıyasla fiyatı oldukça uygundur.
\end{itemize}
   
\begin{itemize}
\item Açık kaynak kodlu olması kullanmak isteyen herkesin özgürlüğine olanak sağlamaktadır. Misal bir eğitim kurumu Arduino için lisans parası ödemeden rahatça kullanabilir. KAYNAK [5],[6]
\end{itemize}


\clearpage