\clearpage
\section{A.3: Kablosuz İletişim }
\label{CH:AltBolum1.3}

Günümüzde kablolu haberleşmenin yanında kablosuz haberleşme de oldukça yaygınlaşmıştır. Bu kablosuz haberleşmeler kullanıcılara herhangi bir yerde herhangi bir anda iletişim kurma olanağı sağlıyor. Kablosuz haberleşme teknolojileri kablo kullanımını 5 kat azaltmaktadır. Bu sistem belli bir mesafede hareket özgürlüğü sağladığından sabit duran veya hareketli bir üründen bilgi almak/göndermek kablosuz haberleşme için büyük bir avantajdır. Tarihte ilk fikir, bilgisayar ve çevresindeki elektronik ürünlerin birbirleriyle kablosuz haberleşmesini sağlamaktı. Daha sonra bu düşünce genişledi ve daha da büyük alanlarda kullanılması hedeflendi. Dar alan kablosuz haberleşme sistemleri özellikle askeri alanda, tıpta, oto sanayide ve AVM'lerde kullanılmaya başlamakta ve gün geçtikçe kullanım alanları artmaktadır. Bu sistemdeki temel amaç, düşük güç harcayarak en az 10 metrelik bir alanda kablo olmaksızın güvenli bir haberleşme ağı kurmaktır. KAYNAK [2]
